\documentclass[10pt, oneside]{article}

% Aesthetics and colors
%-----------------------------------------------
\usepackage[hmargin = 1.25cm, vmargin = 1.5cm]{geometry} 
\usepackage[utf8]{inputenc}  % for the portuguese characters 
\usepackage[none]{hyphenat} % no hyphenated words
\usepackage[usenames,dvipsnames]{xcolor} 
\usepackage{microtype} % to optimize spacing
\usepackage{array}
\usepackage{lmodern}
\usepackage{tikz} % For photo placement
\usepackage{fancyhdr}
\pagestyle{fancy}
\fancyhf{}
\newcommand\tab[1][1cm]{\hspace*{#1}}
\usepackage{multirow}
\usepackage{longtable}

\usepackage{fontawesome}

% Fonts and tweaks for XeLaTeX
%-----------------------------------------------
\usepackage{fontspec,xltxtra}
\usepackage{xltxtra}
\usepackage{xunicode}
\defaultfontfeatures{Mapping=tex-text}
%\setsansfont[Scale=MatchLowercase,Mapping=tex-text]{Gill Sans} % Font my name at the top


%%% Couleurs
\definecolor{baseline-gray}{HTML}{f0f2f2}
\definecolor{baseline-black}{HTML}{212222}
\definecolor{baseline-blue}{HTML}{2d3bff}
\definecolor{baseline-red}{HTML}{ff5043}

% url color and font
%-----------------------------------------------
\usepackage[colorlinks,urlcolor=baseline-black,bookmarks=false,hypertexnames=true]{hyperref}
\urlstyle{same} 

% Customized section headers
%-----------------------------------------------
\usepackage{titlesec} % Allows creating custom \section's
\titleformat{\section}{\bf\color{baseline-black}
	\scshape\Large\raggedright}{}{0em}{}[\color{black}\titlerule]
\titlespacing{\section}{0pt}{0pt}{5pt} % Spacing around titles
\renewcommand{\headrulewidth}{0pt} % Get rid of the default rule in the header

%contact info
\newcommand{\cvemail}{david.beauchemin.5@ulaval.ca}
\newcommand{\cvphonenumber}{(514) 250-3616}
\newcommand{\cvlinkedin}{https://www.linkedin.com/in/david-beauchemin/}
\newcommand{\github}{https://github.com/davebulaval}

\newcommand{\guillemet}[1]{\guillemotleft #1 \guillemotright}

\newcommand{\link}[2]{\href{#1}{#2~{\smaller\faExternalLink*}}}

% Customized subsection headers
%-----------------------------------------------
\titleformat{\subsection}{\bfseries\large\raggedright}{}{0em}{}[\color{black}]
\titlespacing{\subsection}{0pt}{2.5pt}{2.5pt}

% Settings for the vertical timeline
%-----------------------------------------------
\newcolumntype{L}{>{\raggedleft}p{0.14\textwidth}}
\newcolumntype{R}{p{0.8\textwidth}}
\newcommand\VRule{\color{baseline-gray}\vrule width 0.5pt}

\usepackage{fontawesome}

%---------------------------------------------------------
% ----- Begginning of the documment ---------------
%---------------------------------------------------------
\begin{document}
	
	% Name in the top center
	%-----------------------------------------------
	\par{\centering{\bf\sffamily\Huge David Beauchemin}\\\vspace{2pt}
		
		
		% Colored boxes with personal info
		%----------------------------------------------- 
		
		\par{\centering
			\href{mailto:\cvemail}{\faEnvelopeO} \,\, \href{mailto:\cvemail}{\cvemail}\\
			\faPhone \,\, \cvphonenumber \\
			\href{\github}{\faGithub}\,\, \href{\github}{Voir GitHub} \\
			\href{\cvlinkedin}{\faLinkedinSquare}\,\, \href{\cvlinkedin}{Voir profil}\\
		}
		
		\vspace{10pt}
		
		% Formal education
		%-----------------------------------------------
		\section*{Études}
		
		\begin{tabular}{L!{\VRule}R}
			2020 -- 2025 & \textbf{Doctorat en informatique | Apprentissage automatique | Traitement automatique du langage naturel}\\
			\multirow{1}{*}{\includegraphics[scale=0.1]{images/UL_P.pdf} \tab[0.6cm]} &  Université Laval \\
 			& \link{https://graal.ift.ulaval.ca/}{GRAAL - Groupe de recherche en apprentissage automatique de l'Université Laval}\\
			&\\[-6pt]
			
			2018 -- 2020           & \textbf{Maîtrise en informatique | Apprentissage automatique | Traitement automatique du langage naturel}\\
			\multirow{1}{*}{\includegraphics[scale=0.1]{images/UL_P.pdf} \tab[0.6cm]}  &  Université Laval \\
			& \link{https://graal.ift.ulaval.ca/}{GRAAL - Groupe de recherche en apprentissage automatique de l'Université Laval}\\
			& Sujet mémoire: \textit{Détection de doublons parmi des informations non structurées provenant de sources de données externes}\\
			&\\[-6pt]
			
			2015 -- 2018           & \textbf{Baccalauréat en actuariat} \\
			\multirow{1}{*}{\includegraphics[scale=0.1]{images/UL_P.pdf} \tab[0.6cm]}  &  Université Laval
		\end{tabular}
		
		\vspace{10pt}
		
		% Experience
		%-----------------------------------------------
		\section*{Expériences}
		
		\subsection{Professionnelles}
		\begin{tabular}{L!{\VRule}R}
			2026 -- \tab[.7cm] &{\bf {Professionnel de recherche}}\\
			\multirow{1}{*}{\includegraphics[scale=0.1]{images/UL_P.pdf} \tab[0.6cm]}  &  Université Laval, Institut Intelligence et Données \\
			& \\ 
			&\\[-6pt]
			2020 -- \tab[.7cm] &{\bf {Membre fondateur, expert en IA et Directeur général}}\\
			\multirow{1}{*}{\includegraphics[scale=0.075]{images/baseline.png} \tab[1.1cm]}& \href{https://www.baseline.quebec}{Coopérative de solidarité Baseline en intelligence artificielle}\\
			& \\ 
			&\\[-6pt]
			2022 -- 2025&{\bf {Membre étudiant du Conseil de l'Institut Intelligence et Données (IID)}}\\
			\multirow{1}{*}{\includegraphics[scale=0.1]{images/UL_P.pdf} \tab[0.6cm]}  &  Université Laval, Institut Intelligence et Données \\
			&\\[-6pt]
			2020 -- 2022&{\bf {Chargé de projet d'une formation d'environ 200 heures sur l'IA}}\\
			\multirow{1}{*}{\includegraphics[scale=0.1]{images/UL_P.pdf} \tab[0.6cm]}  &  Université Laval, Faculté des sciences et du génie | Formation continue \\
			&\\[-6pt]
			2019 -- 2025 &{\bf {Fondateur, recherchiste, et intervieweur}}\\
			\multirow{1}{*}{\includegraphics[scale=0.025]{images/OpenLayer.png} \tab[0.8cm]}& \link{https://www.youtube.com/channel/UCB3tYpZ1ojiqAroyDN05Cyw}{OpenLayer podcast}\\
			& Création de contenu en IA avec plus de 40 000 vues sur une cinquantaine d'épisodes\\ 
			&\\[-6pt]
			2018 -- 2025   &{\bf {Organisation d'événements}}\\
			\multirow{1}{*}{\includegraphics[scale=0.025]{images/meetup.png} \tab[0.8cm]}& Meetup Machine Learning Québec\\
			& Co-organisation des différents événements à Québec \\ 
			&\\[-6pt]
			2019 -- 2020   &{\bf {Développeur}}\\
			\multirow{1}{*}{\includegraphics[scale=0.15]{images/poutyne.png} \tab[0.18cm]}& \link{https://poutyne.org/}{Poutyne}\\
			& Développement de fonctionnalité \\ 
			&\\[-6pt]
			2019 -- 2020  &{\bf {Scientifique IA}}\\
			\multirow{1}{*}{\includegraphics[scale=0.09]{images/Vooban.png} \tab[1.15cm]}& Vooban\\
			& Conseiller technique en intelligence artificielle
		\end{tabular}
		
		\subsection{Enseignements}
		\begin{tabular}{L!{\VRule}R}
			2022 -- 2024&{\bf {Auxiliaire d'enseignement}}\\
			\multirow{1}{*}{\includegraphics[scale=0.1]{images/UL_P.pdf} \tab[0.6cm]}  &  Université Laval, Département d'informatique et de génie logiciel \\
			& Base de données avancées (GLO-4035/GLO-7035) (2 fois)\\ 
			2016 -- 2022 &{\bf {Auxiliaire de cours}}\\
			\multirow{1}{*}{\includegraphics[scale=0.1]{images/UL_P.pdf} \tab[0.6cm]}  &  Université Laval, École d'actuariat \\
			& Introduction à l'actuariat I, Gestion du risque financier I, Analyse et traitement collectif du risque, Informatique pour Laval University, Méthodes numériques en actuariat, Mathématiques actuarielles IARD II, Modèles linéaires en actuariat, Programmation avec R pour l'analyse de données, Correction des rapports de stage \\ 
			&\\[-6pt]
			2019 -- 2020&{\bf {Auxiliaire d'enseignement}}\\
			\multirow{1}{*}{\includegraphics[scale=0.1]{images/UL_P.pdf} \tab[0.6cm]}  &  Université Laval, CRDM - Centre de recherche en données massives \\
			& École d'hiver en apprentissage automatique
		\end{tabular}
		
		\subsection{Recherches et supervision de stagiaire}
		\begin{tabular}{L!{\VRule}R}
			2021 -- 2025 &{\bf {Auxiliaire de recherche}}\\
			\multirow{1}{*}{\includegraphics[scale=0.1]{images/UL_P.pdf} \tab[0.6cm]}  &  Université Laval\\
			& Supervision de cinq stagiaires pour projet de recherche doctorale \guillemet{Synthèse, reformulation et explication automatiques des contrats d'assurance} \\ 
			& Richard Khoury\\ 
			&\\[-6pt]
			2020 -- 2022 &{\bf {Auxiliaire de recherche}}\\
			\multirow{1}{*}{\includegraphics[scale=0.1]{images/UL_P.pdf} \tab[0.6cm]}  &  Université Laval, IID - Institut intelligence et données\\
			& Supervision de deux stagiaires dans le projet \guillemet{femmes face aux défis de la transformation numérique : une étude de cas dans le secteur des assurances} \\ 
			& Christian Gagné, Centre des compétences futures\\ 
			&\\[-6pt]
			2020 &{\bf {Auxiliaire de recherche}}\\
			\multirow{1}{*}{\includegraphics[scale=0.1]{images/UL_P.pdf} \tab[0.6cm]}  &  Université Laval, École d'actuariat \\
			& Supervision d'un stagiaire et développement du plan d'infrastructure pour le projet de \guillemet{plateforme libre de validation et d'étalonnage de prévisions financières et actuarielles}\\ 
			& Vincent Goulet, Chaire d'actuariat de l'Université Laval\\ 
			&\\[-6pt]
			2019 &{\bf {Auxiliaire de recherche}}\\
			\multirow{1}{*}{\includegraphics[scale=0.1]{images/UL_P.pdf} \tab[0.6cm]}  &  Université Laval, GRAAL - Groupe de recherche en apprentissage automatique de l'Université Laval \\
			& Supervision d'un stagiaire  et développement d'une solution permettant la \guillemet{fusion d'enregistrements de risques commerciaux} \\ 
			& Luc Lamontagne, Chaire de recherche industrielle CRSNG - Intact Corporation financière sur l'apprentissage automatique en assurance\\ 
			&\\[-6pt]
			2019 &{\bf {Auxiliaire de recherche}}\\
			\multirow{1}{*}{\includegraphics[scale=0.1]{images/UL_P.pdf} \tab[0.6cm]}  &  Université Laval, GRAAL - Groupe de recherche en apprentissage automatique de l'Université Laval \\
			& Apprentissage de taxonomie de compétences professionnelles \\ 
			& Luc Lamontagne, Projet ENGAGE\\ 
			&\\[-6pt]
			2018 &{\bf {Auxiliaire de recherche}}\\
			\multirow{1}{*}{\includegraphics[scale=0.1]{images/UL_P.pdf} \tab[0.6cm]}  &  Université Laval, École d'actuariat \\
			& Mesures reliées à l'indexation conditionnelle dans un régime de retraite à prestations déterminées \\ 
			& Louis Adam, Chaire d'actuariat de l'Université Laval\\ 
			&\\[-6pt]
			2018 &{\bf {Auxiliaire de recherche}}\\
			\multirow{1}{*}{\includegraphics[scale=0.1]{images/UL_P.pdf} \tab[0.6cm]}  &  Université Laval, GRAAL - Groupe de recherche en apprentissage automatique de l'Université Laval \\
			& \textit{Weighted bootstrapping method for word relation extraction}  \\ 
			& Luc Lamontagne
		\end{tabular}
	
		\newpage
		
		\section*{Publications}
		\subsection*{\hspace{.5cm} Article}
		\begin{tabular}{L!{\VRule}R}
	2025 & David Beauchemin and Richard Khoury. \link{https://doi.org/10.18653/v1/2025.emnlp-main.6}{\textit{\texttt{QFrCoLA}: a Quebec-French Corpus of Linguistic Acceptability Judgments}}. In \textit{Proceedings of the Conference on Empirical Methods in Natural Language Processing}, pages 119-130, Suzhou, China. Association for Computational Linguistics. \\
	&\\[-6pt] 
	2025 & David Beauchemin, Michelle Albert-Rochette, Richard Khoury, and Pierre-Luc Déziel. \link{https://doi.org/10.18653/v1/2025.emnlp-main.5}{\textit{\texttt{JUDGEBERT}: Assessing Legal Meaning Preservation Between Sentences}}. In \textit{Proceedings of the Conference on Empirical Methods in Natural Language Processing}, pages 92-118, Suzhou, China. Association for Computational Linguistics.\\
	&\\[-6pt] 
	2024 & David Beauchemin, Richard Khoury and Zachary Gagnon. \link{https://doi.org/10.18653/v1/2024.nllp-1.5}{\textit{Quebec Automobile Insurance Question-Answering With Retrieval-Augmented Generation}}. In Proceedings of the Natural Legal Language Processing Workshop at EMNLP, (pp. 48-60).\\
	&\\[-6pt] 
	2023 & David Beauchemin, et Marouane Yassine. \link{https://aclanthology.org/2023.nlposs-1.3/}{\textit{Deepparse: An Extendable, and Fine-Tunable State-Of-The-Art Library for Parsing Multinational Street Addresses}}. Proceedings of the 3rd Workshop for Natural Language Processing Open Source Software (NLP-OSS 2023), pages 19–24, Singapore. Association for Computational Linguistics.\\
	&\\[-6pt] 
	2023 & David Beauchemin, Horacio Saggion, et Richard Khoury. \link{https://www.ncbi.nlm.nih.gov/pmc/articles/PMC10557945/}{\textit{MeaningBERT: assessing meaning preservation between sentences}}. Frontiers in Artificial Intelligence. doi: 10.3389/frai.2023.1223924\\
	&\\[-6pt] 
	2023 & David Beauchemin, et Richard Khoury. \link{https://caiac.pubpub.org/pub/k18zu6c9/release/1}{\textit{RISC: Generating Realistic Synthetic Bilingual Insurance Contract}}. Proceedings of the Canadian Conference on Artificial Intelligence. https://doi.org/10.21428/594757db.132dae7d \textit{(Prix du meilleur article étudiant.)}\\
	&\\[-6pt] 
	2022 & David Beauchemin, et Marie-Claire Monty. \link{https://hal.archives-ouvertes.fr/hal-03736828v2}{\textit{La discrimination en intelligence artificielle est-elle suffisamment encadrée ?}}. Preprint HAL:03736828.\\
	&\\[-6pt] 
	2022 & Vincent Primpied, David Beauchemin, et Richard Khoury. \link{https://caiac.pubpub.org/pub/iaeeogod}{\textit{Quantifying French Document Complexity}}. Proceedings of the Canadian Conference on Artificial Intelligence.\\
	&\\[-6pt] 
	2022 & David Beauchemin, Julien Laumonier, Yvan Le Ster, et Marouane Yassine. \link{https://caiac.pubpub.org/pub/72bhunl6}{\textit{``FIJO'': a French Insurance Soft Skill Detection Dataset}}. Proceedings of the Canadian Conference on Artificial Intelligence.\\
	&\\[-6pt] 
	2021 & Marouane Yassine, David Beauchemin, François Laviolette et Luc Lamontagne. \link{https://arxiv.org/abs/2112.04008}{\textit{Multinational Address Parsing: A Zero-Shot Evaluation}}. Accepted in the International Journal of Information Science et Technology (iJIST)\\
	&\\[-6pt] 	
	2021 & Marouane Yassine, David Beauchemin, François Laviolette et Luc Lamontagne. \link{https://arxiv.org/abs/2006.16152}{\textit{Leveraging Subword Embeddings for Multinational Address Parsing}}. IEEE CiSt, MNLP. \\
	&\\[-6pt] 		
	2020 & David Beauchemin, Nicolas Garneau, Eve Gaumond, Pierre-Luc Déziel, Richard Khoury et Luc Lamontagne. \link{https://arxiv.org/abs/2011.12183}{\textit{Generating Intelligible Plumitifs Summaries: Use Case Application with Ethical Considerations}}. The 13th International Conference on Natural Language Generation. \\
				&\\[-6pt] 
	2020 & David Beauchemin. \link{https://scholar.google.com/scholar?oi=bibs\&cluster=10308625832988857283\&btnI=1\&hl=en\&authuser=1}{\textit{Détection de doublons parmi des informations non structurées provenant de sources de données différentes}}. Mémoire de maitrise, Université Laval. \\
	&\\[-6pt] 
	2019 &  Nicolas Garneau, Mathieu Godbout, David Beauchemin, Audrey Durand et Luc Lamontagne. \link{https://arxiv.org/abs/1912.01706}{\textit{A Robust Self-Learning Method for Fully Unsupervised Cross-Lingual Mappings of Word Embeddings: Making the Method Robustly Reproducible as Well}}. REPORLANG@LREC2020.
		\end{tabular}
		
		\subsection*{\hspace{.5cm} Non scientifique}
		
		\begin{tabular}{L!{\VRule}R}
			2017 & David Beauchemin et Vincent Goulet \textit{\link{https://www.tug.org/TUGboat/Contents/contents38-3.html}{Typesetting actuarial symbols easily and consistently with actuarialsymbol and actuarialangle}}
		\end{tabular}
		
		\subsection*{\hspace{.5cm} Paquetages}
		
		\begin{tabular}{L!{\VRule}R}
			2020 & Marouane Yassine et David Beauchemin \textit{\link{https://deepparse.org}{Deepparse: A state-of-the-art deep learning multinational addresses parser}}\\
			&\\[-6pt] 
			2019 & David Beauchemin \textit{\link{http://notificationdoc.ca/}{Notif: The notification package for every Python project}}\\
			&\\[-6pt] 
			2017 & Simon-Pierre Gadoury et David Beauchemin \textit{\link{https://cran.r-project.org/web/packages/nCopula/index.html}{nCopula: Hierarchical Archimedean Copulas Through Multivariate Compound Distributions}} \\
			&\\[-6pt]  
			2017 & Vincent Goulet et David Beauchemin \textit{\link{http://ctan.org/pkg/actuarialsymbol}{actuarialsymbol - Actuarial notation with \LaTeX}}
		\end{tabular}
	\end{document}